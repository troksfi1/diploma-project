\chapter{Závěr}

\section{Evaluace vlastností Compose Multiplatform ve srovnání s~cíli práce}
Vetšinu cílů, které byly stanoveny v~zadání bylo možné pomocí multiplatformího frameworku Compose Multiplatform splnit a nebylo tak
nutné přistupovat na kompromisy. Veškeré implem

Nicméně při implementacni náročnějších částech aplikace bylo již občas nutné přistoupit k~použití nativních řešení a tím pádem tyto dva přístupy zkombinovat.
Ve výsledku z~poheledu UI, tak tato kombinace nevedla k~zádným ústupkům
Nicméně díký vhodně navrženým principům jako je Expect a actual deklarace (viz paragraf \ref{expectActual}) nebylo skombinování těchto odlišných částí nikterak
implementačně komplikované a ne výsledku ani nepřehledné. 

Fakt že v~některých případech bylo nutné přistoupit k~nativnímu řešení nemusí být brán jakožto nevýhoda Compose Multiplatform oproti ostatním frameworkům.
Na druhou stranu lze tatu možnost vnimat i jako výhodu, kterou ostatní frameworky neposkytují a dělá takz compose Multiplatform z~tohoto pohledu flexibilnější
framework.

prostor o~rozříšení ostatních naticvních knihoven o~mmmultipatformní API. 

Čehož výsledky by v~ideáním případě bylo na daném vývojáří zdali zvolí nativní či multipaltformí cetu vývoje. 

\section{Zkušenosti z~implementace na reálné aplikaci}
Preview
Nedostatek knihoven
Špatná nebo téměř žádna dokumentace k~knihovnám třetích stran a v~případě co se týče Compose Multiplatform

Co se interoperability frameworku Compose Multiplatform s vývojový prostředím Android Studio (verze Jellyfish 2023.3.1), tak zde vývoj doprovázeli 
taktéž problémy v podobě 
Testovani nelze spustit pomocí IDE podobný problém se vyskytoval i v případě testovaní spouštění aplikace pro \textit{desktop}, kterou taktéž pomocí
IDE nebylo možné aplikaci spustit.
%https://stackoverflow.com/questions/78421627/how-can-i-run-tests-in-compose-multiplatform-from-android-studio-command-line
Dále během fáze implementace nebylo možné využít například některé z \code{assertIsNotDisplayed()} jelikož ještě nebyli implementovány.


Běhen fáze návrhu byly obecně známy problémy s nefunkční anotací \textit{@Preview}, která nebyla dostupná a ani aktuálně nefunguje například pro 
\textit{Composable} funkce s parametry. \cite{previewCompose} Důležité je taky zmínit, že 


\section{Závěr o~použitelnosti frameworku}
Jelikož je závěrečné hodnocení této nově nastupující technologie stěžejní částí práce, tak aby bylo toto hodnocení co 
nejobjetivnější, tak bude rozvrženno do několika částí.

Nejprve bude framework porovnán na základě provedné analýzy a to jednak z~pohledu architektury, 

Co se pokrytí aktuálně používaných platforem týka, tak Compose Multiplatform pokrývá většinu z~akluálně nejžádanějších 
platforem pro vývoj multipaltformních aplikací a dá se tak očekávat, že po uvolnění stabilních verzí pro veškeré 
aktuálně nabízené platformy se jeho obliba ještě zvýší. 

V~oblasti výkonu se Compose Multiplatform docela dobře daří a je konkurenceschopný s~ostatními multiplatformními frameworky.
V~případě větších projektů by však tento rozdíl mohl být výraznější.

Rychlost vývoje je také významným hlediskem. 
Co se rychlosti vývoje týká, tak při tvorbě UI, tak V~porovnání s~jinými frameworky, jako je například Flutter, má Compose Multiplatform strukturu kódu s~méně znaky.


V~oblasti optimalizace má však Compose Multiplatform ještě poměrně velké nedostatky a to zejména v~plynulosti provozu na 
platformě iOS.


Pokud jde o~připravenost IDE, situace je obdobná.  Při implementaci a to před
opriti Flutteru nenebízí tolik možnonstí upravy už jako  například zanoření některých kompent do jiných, ale díky jednoduchosti
zápisu UI jsou tyto akce o~něco jednosušší.

A~co se výběru IDE týka, tak z~otestovaných IDE Android Studio jednoznačně předčilo Fleet. Nicméně tato situace se může, 
taktéž rychle změnit.

V~průběhu implementace bylo narazeno na několik problému, které byli nahlášeny a postupně se na nich začína pracovat.

Z~pohledu komunity se s~,


Z~pohledu dokumentace jsou zde velikoé nedostky, ale díky Jetpack Compose není potřeba a většina postupů jak jednotlivé části 
UI tvořit je dobře dokumentovaná společností Google. 

Z~pohledu KMP je tato situace horší
I~přesto, že je KMP již ve stabilní verzi, tak dokumentace oproti ostatním technologiím velmi zaostává.



Následně bude porovnán z~pohledu implementace  jak z~pouheledu tvorby UI, tak z~pohledu tvorby aplikační logiky,



Při závěrečném hodnocení frameworku Compose Multiplatform se dá říci, že 
je Compose Multiplatform vhodným nástrojem pro vývoj multiplatformních aplikací, ačkoli existují oblasti, 
ve kterých může být ještě zlepšen.


Co se týka náročnosti implementace aplikac i na ostatních platformách, tak zde byla náročnost
o~něco zvýšena kvůli nemožnosti implemtace iOS aplikace již od začátku. jelikož je ke kompilaci potřeba 


Podařilo se potvrdit, že množstí sdíleného kódu, které základě provedné analýzy bylo zjištěno je můžné sdílet. V~rámci implementované aplikace
bylo orecento sdíleného kódu ještě větší a to především díky sdílení UI pomocí Compose Multiplatform, ale také možná kvůli menšímu rozsahu
aplikace oproti porovnávaným aplikacím. 


% kapitola main pains
% https://blog.jetbrains.com/kotlin/2021/01/results-of-the-first-kotlin-multiplatform-survey/


\section{Shrnutí dosažených výsledků}
Podařilo se navrhnou aplikaci tak aby fungovala na více platformách.

Součástí návrhu byl také obecně použitelný designový systém, který může být použit k~rozšíření nebo tvorbě nových aplikací v~navrženém stylu.

Aplikace byla úspěšně přizpůsobena konkrétnímu městu a napojena na jeho infrastrukturu pro získávání dat o~událostech, 
aktualitách a parkovacích zónách. Díky implementaci lokální databáze je aplikace použitelná i v~offline režimu.

Díky navržené architektuře je možné aplikaci kdykoliv v~budoucnu rozšířit o~další funkce. 

Poskytuje občanů rychlí přehle o~aktálním dění ve městě a zároveň 

Poskytuje moderní UI, které odpovídá moderním standartům společnost Google pro vývoj mobilních aplikací.
vsetne fukci jako je vyuziti tmavého motivu.

Podařilo se naimplementovat multipaltformí aplikaci, která splňuje požadavky zadání a je spusittelná na mobilních 
platformách Android a iOS. 

Zároveň shlukuje co největší množství kódu ve společné části, čím ukazuje možnosti frameworku a 

zarověň ale ukazaje jak framework umožnuje implementaci navních částí, 

Jedinečný design, pokocí čeho ukažu možnosti daklarativního zápisu UI pomocí Compose Multiplatform.

Z~pohledu UI splnňuje veškeré stanovené požadavky

S~podařilo naimplementovat UI testy, které testují klíčové částí uživatelského rozhraní.

Ukázalo se, že i v~současné fázi vývoje je možné pomocí této technologie vytvořit použitelnou aplikaci.

\section{Zhodnocení splnění cílů práce}
\section{Návrhy pro budoucí vývoj a výzkum}
Z~pohledu aplikace jsou zde velké možnosti pro budoudcí vývoj 
at už z~pohledu rychlosti UI, propojení s~dalšími systémy města, implementaci další UI testu a unit testu
otestovani na vize zařízeních

A~navhnout tak pro města intuitivní aplikace, která by díky multipaltformím vlastnostem pokryla co nejvetší množství zařízení
a zároven by byla snadněji udržovatelná.

Zároveň umožnuje snadnou imtegraci mapových prvků 

nicméně 


Časem se zajisté ukáží

\bigskip

mohou ukázat další možnosti využití tohoto frameworku na jiných platformách

uz je prubehu psani teto prace vzniklo nekolik novych funkcionalit ktere nebyli otesovany jako napriklad moznost
implementovat viepodely na platforme ios stejne jako je tomu na android zarizenich.
