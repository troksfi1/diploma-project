\chapter{Závěr}

\section{Evaluace vlastností Compose Multiplatform ve srovnání s~cíli práce}
Vetšinu cílů, které byly stanoveny v~zadání bylo možné pomocí multiplatformího frameworku Compose Multiplatform splnit a nebylo tak
nutné přistupovat na kompromisy. Veškeré implem

Nicméně při implementacni náročnějších částech aplikace bylo již občas nutné přistoupit k~použití nativních řešení a tím pádem tyto dva přístupy zkombinovat.
Ve výsledku z~poheledu UI, tak tato kombinace nevedla k~zádným ústupkům
Nicméně díký vhodně navrženým principům jako je Expect a actual deklarace (viz paragraf \ref{expectActual}) nebylo skombinování těchto odlišných částí nikterak
implementačně komplikované a ne výsledku ani nepřehledné. 

Fakt že v~některých případech bylo nutné přistoupit k~nativnímu řešení nemusí být brán jakožto nevýhoda Compose Multiplatform oproti ostatním frameworkům.
Na druhou stranu lze tatu možnost vnimat i jako výhodu, kterou ostatní frameworky neposkytují a dělá takz compose Multiplatform z~tohoto pohledu flexibilnější
framework.

prostor o~rozříšení ostatních naticvních knihoven o~mmmultipatformní API. 

Čehož výsledky by v~ideáním případě bylo na daném vývojáří zdali zvolí nativní či multipaltformí cetu vývoje. 

\section{Zkušenosti z~implementace na reálné aplikaci}
Preview
Nedostatek knihoven
Špatná nebo téměř žádna dokumentace ke~knihovnám třetích stran a v~případě co se týče Compose Multiplatform

Co se interoperability frameworku Compose Multiplatform s vývojový prostředím Android Studio (verze Jellyfish 2023.3.1), tak zde vývoj doprovázeli 
taktéž problémy v podobě 
Testovani nelze spustit pomocí IDE podobný problém se vyskytoval i v případě testovaní spouštění aplikace pro \textit{desktop}, kterou taktéž pomocí
IDE nebylo možné aplikaci spustit.
%https://stackoverflow.com/questions/78421627/how-can-i-run-tests-in-compose-multiplatform-from-android-studio-command-line
Dále během fáze implementace nebylo možné využít například některé z \code{assertIsNotDisplayed()} jelikož ještě nebyli implementovány.


Běhen fáze návrhu byly obecně známy problémy s nefunkční anotací \textit{@Preview}, která nebyla dostupná a ani aktuálně nefunguje například pro 
\textit{Composable} funkce s parametry. \cite{previewCompose} Důležité je taky zmínit, že 

\section{Řešení problémů spojených s~multiplatformním vývojem} % mozna presunou do zaveru
Nedostatek knihoven
Mezi jeden z~hlavních problému patří nedostatek vhodných multiplatformních knihoven a ten se projevil i v~případě implementace mapových podkladů
do aplikace kde jak již bylo zmíněno byla nakonec zvolena implementace nativních mapových podkladů za použití expect a actual deklarací (viz paragraf \ref{expectActual}).

Preview
Další problémem bylo nefunkční náhled UI, což se sice na výsledném uživatelském rozhraní nikterak neprojevilo, ale celý proces tvorby UI to značně 
prodloužilo.

Špatná nebo téměř žádná dokumentace k~knihovnám třetích stran a v~případě co se týče Compose Multiplatfrom, tak zde je téměř kompletně spoléháno na
identic. 
Na druhou stranu Compose Multiplatform ještě není na všech platformách ve stabilní verzi a proto je možné, že po vydání stabilních verzí i pro jiné
platformy se dokumentace od JetBrains rozříší.

problémy s~IDE 

\section{Závěr o~použitelnosti frameworku}
Závěrečné zhodnocení bude rozvrženo do několika částí. Nejprve bude framework porovnán na základě provedené analýzy a to jednak 
z~pohledu architektury, tak z TODO


\myparagraph{Podporované platformy}
Co se pokrytí aktuálně podporovaných platforem týká, tak Compose Multiplatform pokrývá většinu z~aktuálně nejžádanějších platforem pro 
vývoj multiplatformních aplikací a dá se tak očekávat, že po uvolnění stabilních verzí pro veškeré aktuálně nabízené platformy se 
jeho obliba ještě zvýší. 


% A protože je platforma Compose Multiplatfom poměrně rýchlá ve svém vývoji v porovnání s jinými frameworky, snadno tak dotahuje 
% svoji konkurenci. Například při tvorbě UI nebo v porovnání s frameworky jako je například Flutter, má Compose Multiplatform strukturu 
% kódu s~méně znaky. 

\myparagraph{Výkon a optimalizace}
V~oblasti výkonu je dle sekce \textit{Porovnání výkonu \ref{performanceSection}} Compose Multiplatform konkurenceschopný s~ostatními 
multiplatformními frameworky, ale kdyby bylo srovnání výkonu provedeno na rozsáhlejších projektech, tak je možné, že by rozdíl mezi 
naměřenými výsledky byl mnohem výraznější.

Co se oblasti optimalizace týká, tak zde má Compose Multiplatform ještě poměrně velké nedostatky a to zejména co se týká plynulosti
provozu na platformě iOS. Pro tu je však Compose Multiplatform stále ve verzi Alfa a z toho důvodu se dá očekávat brzké zlepšení.
%a dá se tak očekávat, že tato oblast bude pro uspěšnost technologie Compose Multiplatform klíčová.

\myparagraph{Optimalizace nástrojů pro vývoj} %DONE
Pokud jde o~připravenost nástrojů pro vývoj, tak mezi hlavními šesti problémy, které dle posledního oficiálního výzkumu KMP
provází, jsou celkem tři, které se optimalizace nástrojů pro vývoj týkají. Konkrétně šlo o problémy s nastavením sestavovaní, problémy s 
vývojovým prostředím a o potíže co se rychlosti sestavování týká. \cite{imgSurvey}. Aktuálně nelze objektivně říci jaké problémy již byly 
vyřešeny a jaké ne, ale v rámci procesu implementace této aplikace byly největší problémy detekovány právě u podpory vývojových prostředí.

Z~testovaných prostředí se vývojové prostředí Android Studio jevilo z pohledu optimalizace lépe optimalizované než konkurenční vývojové 
prostředí Fleet. Na druhou stranu z pohledu funkcí podporuje Fleet některé užitečné funkce jako například \textit{Preview}, které buďto
Android Studio nepodporovalo nebo se u nich vyskytovali zásadní problém, kvůli kterým je nebylo možné použít. Dalším příkladem může být 
například spouštění jednotlivých testů pomocí takzvaných \textit{Gutter Icons}, které pro multiplatformní testy nebylo možné vůbec použít. 

% Nicméně při zmíněné rychlosti vývoje prostředí Fleet se tato situace může taktéž rychle změnit.

% Razantní rozdíl oproti Flutteru je v možnostech jednotlivých úprav. Jedním z příkladů je zanoření určitých komponent do jiných, 
% které v Compose Multiplatform nejsou možné. Benefit v oblasti noření komponet je naopak jednoduchost zápisu UI, které provází 
% celý framework, a proto jsou tyto akce vždy o něco jednodušší.

% zde byla náročnost o~něco zvýšena kvůli nemožnosti implementace 
% iOS aplikace již od začátku. jelikož je ke kompilaci potřeba ? . 

% I když v~průběhu implementace bylo zjištěno několik problémů (jaký?), zjištěné problémy byly nahlášeny a jsou v procesu nápravy.

\myparagraph{Dokumentace}
Z~pohledu dokumentace má Compose Multiplatform významné nedostatky, ale díky tomu, že je z velké části odvozen z frameworku Jetpack Compose,
tak nejsou tyto nedostatky nakonec tak zásadní. Většina potřebných částí pro tvorbu uživatelského rozhraní je totiž dobře dokumentovaná 
společností Google. Co se přidružené technologie KMP týká, tak ta v porovnání s ostatními technologiemi v oblasti dokumentace stejně jako Compose 
Multiplatform zaostává i~přesto, že se již nachází ve stabilní verzi.

\myparagraph{Náročnost implementace}
%Co se týká náročnosti implementace aplikace, tak ta nebýt výše zmíněných problémů, tak implementace probíhala poměrně bezproblémově.

Vyjma výše zminěných problemů, které se naskytly během procesu implementace, celý úkon proběhl téměr bez chyb. Tato bezchybovost a 
jednoduchost se tedy stává velkou výhodou tohoto frameworku. Oproti jiným frameworkům je možnost postupné integrace do již stávajících 
aplikací, která v aktuální fázi přecházení od imperativních způsobů zápisu UI k deklarativním způsobům může tomuto frameworku pomoct 
nabýt na popularitě. 


\section*{Shrnutí}
Při závěrečném hodnocení frameworku Compose Multiplatform se dá říci, že je Compose Multiplatform vhodným nástrojem pro vývoj multiplatformních aplikací, i když existují oblasti, ve kterých může být ještě zlepšen.


\section{Shrnutí dosažených výsledků}
Co se vytvořené aplikace týká, tak ta byla úspěšně přizpůsobena konkrétnímu městu a napojena na jeho infrastrukturu pro získávání dat 
o~událostech, aktualitách a parkovacích zónách. Díky implementaci lokální databáze je navíc aplikace použitelná i v~offline režimu, čímž 
usnadňuje přístup všem uživatelům a urychluje tak přehled o~aktuálním dění ve městě. Navržená architektura je zároveň dostatečně flexibilní, 
aby pokryla případně změny ve struktuře města.

Co se frameworku Compose Multiplatform týká, tak zde se podařilo na základě prakticky sestavené aplikace potvrdit, že aplikace je schopná 
sdílení určitého množství kódu. Přímo při implementaci aplikace bylo procento sdíleného kódu násobně větší oproti dosavadním aplikacím
používaných v praxi. Množství sdíleného kódu se zvýšilo především díky sdílení UI pomocí Compose Multiplatform. 
V rámci výsledků je ale nutné zohlednit fakt, že rozsah zkušební aplikace nedosahuje 
takových rozměrů oproti ostatním porovnávaným aplikacím. Každopádně lze konstatovat, že se i v~současné fázi vývoje se podařilo navrhnout
aplikaci tak, aby fungovala na více platformách a zároveň je možné pomocí této technologie vytvořit použitelnou aplikaci. Z~pohledu UI 
splnňuje veškeré stanovené požadavky, které mohou být pro aplikaci menšího rozměru dostačující. 

\bigskip






%\section{Zhodnocení splnění cílů práce}
\section{Návrhy pro budoucí vývoj a výzkum}
Z~pohledu aplikace jsou zde velké možnosti pro budoucí vývoj a to ať už z~pohledu rychlosti UI, propojení s~dalšími systémy města, 
implementaci dalších UI testů a unit testů na více zařízeních. Výsledkem by tak mohla být obecně použitelná aplikace pro města,
která by díky multiplatformním vlastnostem pokryla co největší množství zařízení a zároveň by byla snadněji a levněji udržovatelná.

Při rychlosti vývoje toho frameworku se jistě brzy ukáží další možnosti využití.
Už průbéhu psaní této práce vzniklo několik nových funkcionalit, které nebyly otestovány jako například možnost
implementovat Viewmodely na platformě iOS, stejně jako je tomu na android zařízeních.

%Zároveň se zde nabízí rozšíření aplikace i na další podporované platformy 



