% Do not forget to include Introduction
%---------------------------------------------------------------
\chapter{Úvod}
% uncomment the following line to create an unnumbered chapter
%\chapter*{Úvod}\addcontentsline{toc}{chapter}{Introduction}\markboth{Introduction}{Introduction}
%---------------------------------------------------------------
\setcounter{page}{1}

% The following environment can be used as a mini-introduction for a chapter. Use that any way it pleases you (or comment it out). It can contain, for instance, a summary of the chapter. Or, there can be a quotation.
%\begin{chapterabstract}
%	\lipsum[1]
%\end{chapterabstract}

%\section{Motivace}
\section{Cíle práce} \label{goals}
Prvním důležitým cílem práce je analyzovat použitelnost nově nastupující technologie na poli vývoje multiplatformního UI a otestovat, jaké
nové principy přináší. Dále tuto technologii porovnat s~ostatními frameworky pro tvorbu multiplatformního UI a pokusit se
rozebrat důležité principy, na kterých jsou tyto technologie postaveny. 

Dalším podstatným cílem je vytvoření aplikace, která tuto technologii implementuje a zaměření se při tom na slabé stránky, 
které aktuálně multiplatformní UI provází. 
Posledním hlavním cílem je vytvořenou aplikaci otestovat a zhodnotit použitelnost daného frameworku v~praxi.

\section{Stručný přehled obsahu práce}
V~úvodu se práce zabývá otázkou, co to vlastně multiplatformní vývoj je, proč se jím zabývat a jakých zařízení se může týkat.
Dále je probrána otázka v~jakých případech se vyplatí multiplatformní UI aplikovat a na jakých zařízeních. 

%a porovnání vazby s UI s logikou daného aplikace vůči multipaltfomovodsti.



%kdy multiplatformí vývoj nemá smysl (vyfocení závady)


Jaké jsou výhody a nevýhody multiplatformních aplikací  v~porovnaní s~nativními aplikacemi.

Existujícími frameworky a jich porovnáním s~Compose Multiplatform.

Detailně rozebrána architektura Compose Multiplatform, jak framework funguje a jaké další technologie jsou potřeba k~smysluplné implementaci
 tohoto frameworku.

Následně se práce zabývá návrhem a následnou tvorbou multiplatformního UI pomocí frameworku Compose Multiplatform.

V~jedné z~posledních částí se práce věnuje možnostem testování takto vytvořeného UI a následného testování UI na naimplementované aplikaci.

je věnován shrnutí celého procesu implementace a vyzdvižení naskytlých problémů a výhod oproti nativnímu případně jiným multiplatformním frameworkům


\section{Definice multiplatformního vývoje UI}

Technologie sloužící k~tvorbě multiplatformních UI umožňují vývojářům vytvářet jednotná uživatelská rozhraní, 
která mohou být nasazena na různá zařízení jako jsou mobilní zařízení, tablety, televize, hodinky, obrazovky aut či 
klasické počítače a to buďto v~podobě desktopové nebo webové aplikace. Cílem multiplatformního vývoje je tak dosáhnout jednotného 
uživatelského zážitku bez nutnosti psát a udržovat oddělené kódy pro každou platformu. Nicméně, jak je vidno z~předchozího výčtu zařízení, 
tak ne vždy má smysl multiplatformní přístup aplikovat. Tomu, kdy je vhodné aplikovat multiplatformní vývoj, je věnována následující kapitola.  

\section{Uplatnění multiplatformního vývoje a sdíleného UI}
Aktuálně největší uplatnění multiplatformního UI nabízí mobilní vývoj, kde je díky multiplatformním frameworkům možné vyvíjet
aplikace pro platformy Android a iOS současně. Je tomu tak z~důvodu, že mobilní vývoj je aktuálně jeden TODO také díky podobnostem majících 
na návrh multiplatformního UI vliv jako je například velikost obrazovky.

I~přesto, že multiplatformní frameworky umožňují implementaci multiplatformním UI na většinu nejvíce používaných platforem jako je Android, iOS,
Android TV, tvOS, Wear OS, watchOS nebo Android Automotive, tak né vždy je vhodné těchto možností využít. Z~pohledu UI mají ty platformy často jiné 
velikosti obrazovek, jiné možnosti ovládaní a proto, je často multiplatformní UI a často i veškerá logika implementována konkrétně pro danou platformu
nebo typ zařízení. Je proto vždy nutné vědět jaký typ aplikace bude z~pohledu aplikační logiky implementován, jaké budou jeho způsoby užití a na jakých 
platformách bude daná aplikace implementována.

Obecně lze tedy říci, že efektivita implementace multiplatformních aplikaci se může dle těchto omezeních a požadavků výrazně lišit.

%I přesto, že tato práce je věnována především tvorbě UI, tak nesmí být opomenuta aplikační logika, které se za ním skrývá a má UI podstatný vliv.
% Dále zde hrají velkou roli případy užití, které mají na jednotlivá a UI vliv a při multiplatformním vývoji je třeba dbát na jelikož né každá 
% aplikace je z tohoto pohledu vhodná pro multiplatformní

\section{Důvody multiplatformního vývoje a sdíleného UI}

Mezi primární důvody vedoucí firmy a jednotlivce k~vývoji multiplatformních aplikací patří především
snížení nákladů na vývoj a následně také na údržbu. Jednodušší dosažení konzistentního vzhledu
na různých platformách nebo například možnost znovu používat komponenty UI na různých platformách.

Mezi další důvody může například patřit možnost rychlejších aktualizací, jelikož nové funkce mohou být 
implementovány jednotně a rychle na všech platformách. 