% Do not forget to include Introduction
%---------------------------------------------------------------
\chapter{Úvod}
% uncomment the following line to create an unnumbered chapter
%\chapter*{Úvod}\addcontentsline{toc}{chapter}{Introduction}\markboth{Introduction}{Introduction}
%---------------------------------------------------------------
\setcounter{page}{1}

% The following environment can be used as a mini-introduction for a chapter. Use that any way it pleases you (or comment it out). It can contain, for instance, a summary of the chapter. Or, there can be a quotation.
%\begin{chapterabstract}
%	\lipsum[1]
%\end{chapterabstract}

%\section{Motivace}

V rámci úvodní kapitoly se práce zabývá otázkami, co to vlastně multiplatformní vývoj je, proč se jím zabývat a jakých zařízení se může týkat.
Dále je probrána otázka, v~jakých případech se vyplatí multiplatformní uživatelské rozhraní (UI) aplikovat a na jakých konkrétních zařízeních. 

%a porovnání vazby s UI s logikou daného aplikace vůči multipaltfomovodsti.



%kdy multiplatformí vývoj nemá smysl (vyfocení závady)

V rámci analýzy jsou detailně představeny aktuálně používané frameworky pro tvorbu multiplatformních aplikací včetně samotného
frameworku Compose Multiplatform a jeho vazby na technologii Kotlin Multiplatform. Obě tyto technologie jsou následně podrobně rozebrány
a nakonec porovnány s aktuálně používanými technologiemi z hlediska výkonu a velikosti samotných aplikací.

\medskip

Druhá kapitola se věnuje výběru a návrhu aplikace, která slouží k otestovaní použitelnosti technologie Compose Multiplatform.
Při fázi návrhu je nejprve přistoupeno k vytyčení případů užití a následně jsou specifikovány funkční a nefunkční požadavky pro vybranou aplikaci.
Pro aplikaci je navržena základní architektura, která nejlépe vyhovuje vytyčeným požadavkům a následně práce přechází k detailnímu
návrhu samotného UI. To je nejprve navrženo pomocí takzvaných drátěných modelů a následně převedeno na takzvané mockup modely, které
lépe vystihují výslednou grafickou podobu UI.

\medskip

Kapitola implementace popisuje založení multiplatformního projektu pomocí frameworku Compose Multiplatform
a implementaci navrženého uživatelského rozhraní pomocí této technologie. Dále se tato kapitola věnuje kritickým UI komponentám
jako je použitá navigace a následně také důležitým technologiím použitých pro správné fungování aplikační logiky. Závěr této kapitoly
je pak věnován implementované logice uživatelského rozhraní.

\medskip

Předposlední kapitolou je kapitola věnují se testovaní implementované aplikace, v rámci které jsou nejprve představeny možnosti testování
multiplatformního UI, následně jsou tyto testy navrženy a implementovány.


\medskip


Poslední kapitola je věnována shrnutí celého procesu implementace a konkretizaci naskytlých problémů. Úplně poslední část této kapitoly se týká celkové evaluace
vlastností frameworku Compose Multiplatform a závěrečného zhodnocení použitelnosti toho frameworku v praxi.

\newpage

\section{Cíle práce} \label{goals}
Prvním důležitým cílem práce je zanalyzovat použitelnost nově nastupující technologie na poli vývoje multiplatformního UI a otestovat, jaké
nové principy přináší. Dále tuto technologii porovnat s~ostatními frameworky pro tvorbu multiplatformního UI a pokusit se
rozebrat důležité principy, na kterých jsou tyto technologie postaveny. 

Dalším podstatným cílem je vytvoření aplikace, která tuto technologii implementuje a zaměření se při tom na slabé stránky, 
které aktuálně multiplatformní UI provází. 
Posledním hlavním cílem je vytvořenou aplikaci otestovat a zhodnotit použitelnost daného frameworku v~praxi.

\begin{sloppypar}
\section{Definice multiplatformního vývoje UI}
Technologie sloužící k~tvorbě multiplatformních UI umožňují vývojářům vytvářet jednotná uživatelská rozhraní, 
která mohou být nasazena na různá zařízení jako jsou mobilní zařízení, tablety, televize, hodinky, obrazovky aut či 
klasické počítače a to buďto v~podobě desktopové nebo webové aplikace. Cílem multiplatformního vývoje je tak dosáhnout jednotného 
uživatelského zážitku bez nutnosti psát a udržovat oddělené kódy pro každou platformu. Nicméně, jak je vidno z~předchozího výčtu zařízení, 
tak ne vždy má smysl multiplatformní přístup aplikovat. Tomu, kdy je vhodné aplikovat multiplatformní vývoj, je věnována následující sekce.  
\end{sloppypar}
% Aktuálně největší uplatnění multiplatformního UI nabízí mobilní vývoj, kde je díky multiplatformním frameworkům možné vyvíjet
% aplikace pro platformy Android a iOS současně. Je tomu tak z~důvodu, že mobilní vývoj je aktuálně jeden TODO také díky podobnostem majících 
% na návrh multiplatformního UI vliv jako je například velikost obrazovky.


\section{Uplatnění multiplatformního vývoje a sdíleného UI}
I~přesto, že multiplatformní frameworky umožňují implementaci multiplatformního UI na většinu nejvíce používaných platforem jako je Android, iOS,
Android TV, tvOS, Wear OS, watchOS nebo Android Automotive, tak ne vždy je vhodné těchto možností využít. Z~pohledu UI mají tyto platformy často jiné 
velikosti obrazovek nebo jiné možnosti ovládaní a proto, je v některých případech nutné implementovat UI a často i veškerou logiku pro konkrétní platformu
nebo typ zařízení. Je proto vždy nutné vědět, jaký typ aplikace bude z~pohledu aplikační logiky implementován, jaké budou jeho způsoby užití a na jakých 
platformách bude daná aplikace implementována.
Obecně lze tedy říci, že efektivita implementace multiplatformních aplikací se může dle těchto omezení a požadavků výrazně lišit.

%I přesto, že tato práce je věnována především tvorbě UI, tak nesmí být opomenuta aplikační logika, které se za ním skrývá a má UI podstatný vliv.
% Dále zde hrají velkou roli případy užití, které mají na jednotlivá a UI vliv a při multiplatformním vývoji je třeba dbát na jelikož né každá 
% aplikace je z tohoto pohledu vhodná pro multiplatformní

\section{Důvody multiplatformního vývoje a sdíleného UI}

Mezi primární důvody vedoucí firmy a jednotlivce k~vývoji multiplatformních aplikací patří především
snížení nákladů na vývoj a následně také na údržbu. \cite{crossPlatformFrameworks} Dále také jednodušší dosažení konzistentního vzhledu
na odlišných platformách nebo například možnost znovu používat komponenty UI na různých platformách.

Mezi další důvody může například patřit možnost rychlejších aktualizací, jelikož nové funkce mohou být 
implementovány jednotně a rychle na všech platformách. 
