% Do not forget to include Introduction
%---------------------------------------------------------------
\chapter{Úvod}
% uncomment the following line to create an unnumbered chapter
%\chapter*{Úvod}\addcontentsline{toc}{chapter}{Introduction}\markboth{Introduction}{Introduction}
%---------------------------------------------------------------
\setcounter{page}{1}

% The following environment can be used as a mini-introduction for a chapter. Use that any way it pleases you (or comment it out). It can contain, for instance, a summary of the chapter. Or, there can be a quotation.
%\begin{chapterabstract}
%	\lipsum[1]
%\end{chapterabstract}

%\section{Motivace}
\section{Cíle práce}
Prvním důležitým cílem práce je analyzovat použitelnost nově nastupující technologie na poli vývoje multiplatformního UI a otestovat, jaké
nové principy přináší. Dále tuto technologii porovnat s ostatními frameworky pro tvorbu multiplatformního UI a pokusit se
rozebrat důležité principy na, kterých jsou tyto technologie postaveny. 

Další podstatným cílem je vytvoření aplikace, která tuto technologii používá a zaměření se při tom na slabé stránky, 
které aktuálně multiplatformní UI provází.

Posledním hlavním cílem je vytvořenou aplikaci otestovat a zhodnotit použitelnost daného frameworku v praxi.

\section{Stručný přehled obsahu práce}

Obecně něco o multiplatformním vývoji.

%---------------------------------------------------------------
\section{Teoretický úvod}
%---------------------------------------------------------------

\section{Definice multiplatformního vývoje UI}

Technologie sloužících k tvorbě multiplatformních UI umožňují vývojářům vytvářet jednotná uživatelské rozhraní, 
která mohou být nasazena na různé platformy, jako jsou mobilní zařízení, webové prohlížeče nebo desktopové aplikace. 
Cílem je dosáhnout jednotného uživatelského zážitku bez nutnosti psát a udržovat oddělené kódy pro každou platformu.


\section{Význam multiplatformního vývoje a sdíleného UI}

Mezi primární důvody vedoucí firmy a jednotlivce k vývoji multiplatformních aplikací patří především
snížení nákladů na vývoj a následně také na údržbu. Jednodušší dosažení konzistentního vzhledu
na různých platformách nebo například možnost znovu používat komponenty UI na různých platformách.

Mezi další důvody může například patřit možnost rychlejších aktualizací, jelikož nové funkce mohou být 
implementovány jednotně a rychle na všech platformách. 



%Větší dosah jelikož díky sdílení UI lze dosáhnout většího dosahu, protože aplikace mohou být dostupné 

\chapter{Analýza}

\section{Přehled existujících frameworků}
Mezi aktuálně nejpopulárnější multiplatformní frameworky jednoznačně patří Flutter a React Native. \cite{crossPlatformFrameworksStats}
V následujících kapitolách jsou proto tyto nejpoužívanější frameworky podrobněji rozebrány a u každého z nich jsou vybrány 
důležité vlastnosti, které jsou pro tyto frameworky typické. Jednotlivé frameworky jsou seřazeny postupně od 
pro tvorbu multiplatformního UI existují, jak se postupem času vyvíjely a jaké mají společné rysy.

%---------------------------------------------------------------
\subsection{React Native}
%---------------------------------------------------------------
React Native byl jedním z prvních frameworků pro tvorbu multiplatformního UI a do jisté míry ovlivnil i ostatní popisované
frameworky. Jeho vývoj započal ve společnosti Facebook během interního hackaton projektu a první jeho oficiálně publikovaná
verze vyšla začátkem roku 2015. \cite{reactNativeHistory}
Nyní se jedná o open-source framework, kde jeho hlavním cílem je umožnit vývojářům vytvářet nativní mobilní aplikace 
pro platformy Android a iOS z jednoho společného kódu napsaného v jazyce JavaScript nebo TypeScript.

\section*{Klíčové vlastnosti React Native}

\myparagraph{Komponentní architektura} 
React Native využívá komponentní architekturu, která vývojářům umožňuje 
vytvářet znovupoužitelné komponenty. \cite{reactNativeComponents} Tato architektura je jednak založena na obecných
React komponentách, ale zároveň také na React Native specifických komponentách, které se dále dělí na takzvané core
komponenty, komponenty vytvořené komunitou či vlastní nativní komponenty. \cite{reactNativeComponents}
    
\myparagraph{Deklarativní zápis UI}
React Native stejně jako React pro webové aplikace využívá pro zápis UI deklarativní způsob, 
při kterém využívá JSX (JavaScript XML) syntaxe k popisu struktury UI komponent. \cite{reactNativeJSX}
Takto zapsané UI lépe reflektovalo aktuální stav aplikace.
Tento způsob zápisu začal na mobilních platformách růst popularitě právě díky Reactu Native, který po svém uvolnění v roce 2013 
defacto nastartoval éru deklarativního zápisu UI na mobilních platformách. \cite{declarativeUIHistory}

\myparagraph{Fast Refresh} 
Fast Refresh je v funkce, která vývojářům umožňuje okamžitě vidět výsledky provedených 
změn v kódu bez nutnosti znovu sestavení aplikace. \cite{reactNativeFastRefresh}

\myparagraph{JavaScript/TypeScript} 
Aplikační logika v React Native se píše v jazyce JavaScript nebo TypeScript, 
což usnadňuje snadnou integraci s existujícími webovými technologiemi. \cite{reactNativeFundamentals}

\myparagraph{Rozsáhlá komunita} 
React Native má rozsáhlou komunitu vývojářů, což vede k bohatému ekosystému 
třetích stran, včetně mnoha dostupných knihoven a modulů. \cite{reactNativeComunity}

\myparagraph{Expo framework} 
Pro ještě snazší start vývoje poskytuje React Native Expo framework, který 
zjednodušuje proces vývoje a umožňuje rychlé prototypování. \cite{reactNativeExpo}

\subsection*{Architektura frameworku React Native}

React native do roku 2022 využíval architekturu založenou na návrhového vzoru Bridge, který spojoval kód napsaný v JavaScriptu s nativní kódem určeným pro danou platformu. 
Tyto dva celky byly spuštěny na souběžných vláknech a komunikovaly spolu pomocí zasílání serializovaných zpráv. Časem se ale ukázalo, že se tato komunikace
a další její charakteristiky stávají úzkým hrdlem celého systému a byla proto od verze 0.68 nahrazena JavaScriptovým rozhraním zvaným JSI. \cite{reactNativeAboutNewArch}

JSI nově JavaScriptu umožňuje držet referenci na C++ objekty a volat nad nimi potřebné metody. \cite{reactNativeAboutNewArch} Toho využívá nový renderovací systém zvaný \textit{Fabric}, který 
frameworku napomáhá sjednotit renderovací logiku prováděnou v C++ a zlepšit tak interoperabilitu s nativními platformami.

\begin{figure}[H]
  \centering
  \includegraphics[width=0.85\textwidth]{react-natice-xplat-implementation-diagram.png}
  \caption{React Native }
  \label{fig:react-natice-xplat-implementation-diagram}
\end{figure}

% \begin{figure}[H]
%   \centering
%   \includegraphics[width=0.55\textwidth]{react-native-data-flow.jpg}
%   \caption{React Native tok dat}
%   \label{fig:react-natice-data-flow}
% \end{figure}


% I přes tyto změny React Native stále využívá systém dvou typů vláken označovaných jako \textit{JavaScript Thread} a \textit{UI Thread} (známé taktéž pod názvem \textit{Main Thread}) 
% mezi které rozděluje práci.


Jak je vidět na obrázku \ref{fig:react-natice-xplat-implementation-diagram}, tak Fabric de facto plní funkci prostředníka, který převádí React 
komponenty na nativní komponenty pro každou platformu. 

Pro lepší představu jak dané komponenty vypadají slouží následující ukázka kódu \ref{lst:reactNativeJSX}.

\begin{lstlisting}[caption={Popis UI komponent pomoci JSX}, label={lst:reactNativeJSX}, language=XML]
function MyComponent() {
  return (
    <View>
      <View
        style={{backgroundColor: 'red', height: 20, width: 20}}
      />
      <View
        style={{backgroundColor: 'blue', height: 20, width: 20}}
      />
    </View>
  );
}
// <MyComponent />
\end{lstlisting}

Na ukázce kódu \ref{lst:reactNativeJSX} jsou použity některé z nejpoužívanějších core komponent, které se při psaní React
Native aplikací používají. \cite{reactNativeComponents} Jak již bylo zmíněno dříve, tak tyto komponenty jsou následně převedeny 
na nativní komponenty pro každou platformu a to jakým způsobem Fabric dané komponenty převádí se dá rozdělit do následujících tří
na sebe navazujících fází: \cite{reactNativeRenderCommitMount}
 
%Na této ukázce je zároveň vidět použití deklarativního zápisu, které bude podrobněji probráno v následující kapitole.

\smallskip

\myparagraph{Render}
Během této fáze se z jednotlivých komponent (React Elementů) sestaví strom elementů v JavaScriptu (viz levá část obrázku \ref{fig:react-native-render-pipeline}) a nad tímto stromem se 
následně spustí rekurzivní redukce, při které dojde k vytvoření nového stromu takzvaného React Shadow Tree. (viz prostřední část obrázku \ref{fig:react-native-render-pipeline})
\cite{reactNativeRenderCommitMount} Ten se skládá z jednotlivých React Shadow Nodes, které reprezentují objekty v C++ a tím přechází tato renderovací fáze do další fáze zvané commit.\cite{reactNativeRenderCommitMount}

\myparagraph{Commit}
V rámci této fáze je pro každý React Shadow Node vypočítána jeho pozice a velikost na koncovém zařízení a díky tomu může renderovací systém
přejít k poslední fázi zvané Mount. \cite{reactNativeRenderCommitMount}

\myparagraph{Mount}
Během této poslední fáze dojde k transformaci \textit{React Shadow Tree} na \textit{Host View Tree} (viz pravá část obrázku \ref{fig:react-native-render-pipeline}) a to tak, že každý \textit{React Shadow Node}
se transformuje na jeho ekvivalent v nativní podobě. \cite{reactNativeRenderCommitMount} Čili například na platformě Android se $<ViewShadowNode>$ přetransformuje na android.view.ViewGroup. \cite{reactNativeRenderCommitMount}

\begin{figure}[H]
  \centering
  \includegraphics[width=0.99\textwidth]{react-native-render-pipeline.png}
  \caption{React Native vykreslovací fáze}
  \label{fig:react-native-render-pipeline}
\end{figure}

%---------------------------------------------------------------
\subsection{Flutter}
%---------------------------------------------------------------
Flutter je open-source softwarový toolkit pro vývoj uživatelských rozhraní (UI). \cite{flutterfaq} Za vývojem stojí společnost Google a je určený k vytváření nativně kompilovaných 
aplikací pro mobilní zařízení, web a desktop z jednoho zdrojového kódu. \cite{flutterfaq}
Byl vydán v roce 2017 a získal značnou popularitu mezi vývojáři díky svému snadnému použití, flexibilitě a schopnostem tvorby UI.

\subsection*{Klíčové vlastnosti Flutteru}

\myparagraph{UI založené na widgetech} 
Flutter využívá reaktivně deklarativní UI založené na widgetech. \cite{flutterUI} Widgety jsou základními stavebními 
bloky Flutter aplikací, představující vše od strukturálních prvků po stylistické komponenty. \cite{flutterWidgets}

\myparagraph{Hot Reload} 
Další z významných funkcí Flutteru, která významně zrychluje vývojový proces a zvyšuje produktivitu je funkce "Hot Reload". 
Během vývoje běží Flutter aplikace na virtuálním počítači (Dart VM), který díky této funkci umožňuje okamžitě vidět provedené změny
v kódu bez nutnosti úplné rekompilace aplikace. \cite{flutterHotReload} Teprve pro vydání jsou Flutter aplikace kompilovány přímo do strojového kódu, 
ať už jde o instrukce Intel x64 nebo ARM, případně do JavaScriptu pokud jsou cíleny na web. \cite{flutterArchOverview}


\myparagraph{Jeden zdrojový kód pro více platforem} 
S Flutterem mohou vývojáři psát jeden zdrojový kód pro obě platformy Android a iOS, což snižuje dobu vývoje a úsilí 
vynakládané na údržbu. Flutter také rozšířil svou podporu pro cílení webových a desktopových aplikací, umožňující širší dosah s minimálními změnami kódu. \cite{flutter}

\myparagraph{Rozsáhlá sada widgetů}
Flutter poskytuje komplexní sadu přizpůsobitelných widgetů, které usnadňují vytváření složitých a vizuálně 
atraktivních uživatelských rozhraní. Tyto widgety zahrnují vše od základních tlačítek a textových polí až po 
pokročilé komponenty jako jsou grafy a animace.

\myparagraph{Programovací jazyk Dart}
Aplikace vytvořené ve Flutteru jsou psány v jazyce Dart, moderním objektově orientovaném programovacím jazyce 
vyvinutém společností Google. Dart je navržen pro optimální výkon a produktivitu, což ho činí vhodným pro mobilní a
webový vývoj. \cite{dart}

\myparagraph{Způsob renderovaní UI}
Narozdíl od Reactu Native Flutter nepoužívá platformě specifické komponenty koncových zařízení, ale veškeré UI komponenty (widgety)
renderuje pomocí vlastního renderovacího enginu. \cite{flutterRenderingModel}


\subsection*{Architektura frameworku Flutter} 
Jak je vidět na obrázku \ref{fig:flutter_architectural_layers}, tak architektura Flutteru je rozdělena do tří hlavních vrstev. \cite{flutterArchOverview}
Embedder je vrstva, která umožnuje integrovat Flutter do konkrétních platforem jako je Android, iOS, desktop nebo web. 
Každý embedder obsahuje platformě specifický kód, který je potřebný pro spuštění Flutteru na dané platformě.
Engine je vrstva starající se o vykreslování grafiky, kdykoli kdy je potřeba vykreslit nový snímek. \cite{flutterArchOverview} 
Framework je vrstva, která je používána vývojáři k vytváření uživatelských rozhraní a definování chování aplikace. 
Obsahuje hotové widgety, funkce pro manipulaci s UI a další nástroje pro vývojáře. \cite{flutterArchOverview} 


\begin{figure}[H]
  \centering
  \includegraphics[width=0.85\textwidth]{flutter_architectural_layers.png}
  \caption{Flutter architectural layers}
  \label{fig:flutter_architectural_layers}
\end{figure}

Mezi hierarchicky poslední a neméně důležitou častí tohoto frameworku jsou platformově specifické knihovny Material and Cupertino. Tyto knihovny
jsou následně využívány widgety k implementaci konkrétního design systému. Díky tomu je možné uživateli navodit nativní pocit z dané aplikace.

\medskip

Z pohledu UI je důležitým prvkem právě Flutter framework, který zároveň definuje jak spolu jednotlivé widgety interagují.

Widget je ve Flutteru základní stavební blok pro tvorbu uživatelského rozhraní. \cite{flutterWidgets} V následující ukázce kódu \ref{lst:flutterCode} je pro příklad použito několik 
základních widgetů jako je \emph{Image} nebo \emph{Text} a taktéž layout widget zvaný \emph{Container} a \emph{Row} pro organizaci a 
rozložení vnořených widgetů na obrazovce.

\begin{lstlisting}[caption={Popis UI widgetů pomocí jazyka Dart}, label={lst:flutterCode}, language=Kotlin]
  Container(
  color: Colors.blue,
  child: Row(
    children: [
      Image.network('https://www.example.com/1.png'),
      const Text('A'),
    ],
  ),
);
\end{lstlisting}

Když Flutter potřebuje vykreslit tento blok, zavolá metodu \emph{build()} a ta vrátí podstrom widgetů, které následně vykreslí uživatelské 
rozhraní na základě aktuálního stavu aplikace. \cite*{flutterArchOverview}

Během fáze sestavování překládá Flutter widgety vyjádřené v kódu (například kód \ref{lst:flutterCode}) do odpovídajícího stromu elementů viz obrázek \ref{fig:flutter_trees}, přičemž každý widget má jeden element a 
každý prvek představuje určitou instanci widgetu v daném umístění stromové hierarchie. \cite*{flutterArchOverview}


\begin{figure}[H]
  \centering
  \includegraphics[width=1\textwidth]{flutter_trees.png}
  \caption{Flutter build proces}
  \label{fig:flutter_trees}
\end{figure}



\subsection{Compose Multiplatform}

Compose Multiplatform je framework sloužící k tvorbě uživatelských rozhraní použitelných na vícero platformách za 
jehož vývojem stojí společnost JetBrains. \cite{composeMultiplatform} Je založen na toolkitu zvaném Jetpack Compose, který je aktuálně 
doporučovaný k tvorbě nativních uživatelských rozhraních na platformě Android. \cite{jetpack}

Podporuje platformy jako Android, iOS (Alpha), Windows, MacOS, Linux a Web (experimentální). \cite{composeMultiplatform}

\medskip

\subsection*{Klíčové vlastnosti Compose Multiplatform}

\myparagraph{Deklarativní zápis UI} 
Používá deklarativní syntaxi pro popis uživatelského rozhraní. \cite{KMPUseCases}

\myparagraph{Jednotný kód pro různé platformy} 
Možnost sdílet kód pro Android, iOS, web i Desktop.

\myparagraph{Snadné migrace díky postupné integraci} 
Díky KMP je možné postupně implementovat jednotlivé částí aplikace i do již existujících aplikací s minimálním rizikem oproti
ostatním multiplatformním technologiím. \cite{KMP}

\myparagraph{Znovupoužitelnost Kotlin kódu} 
Díky KMP je možné použít některé části kódu z již existujících Android aplikací i na ostatních platformách.

\myparagraph{Programovací jazyk Kotlin} 
Frontendová i backendová část aplikace jsou psány v jazyce Kotlin pro bezproblémovou integraci se serverovou částí.

\myparagraph{Podpora od JetBrains}
Poskytuje stabilní podporu od vývojářského týmu JetBrains.

\subsubsection{Kotlin Multiplatform}


%https://blog.jetbrains.com/kotlin/2023/11/kotlin-multiplatform-stable/#use-the-power-of-the-growing-kotlin-multiplatform-ecosystem

Kotlin Multiplatform je často základním kamenem pro tvorbu multiplatformních aplikací založených na Compose Multiplatform.
Od listopadu 2023 má svoji stabilní verzi, která je stoprocentně připravena na produkční nasazení. \cite{KMPstable}

Jelikož se jedná o SDK, tak umožňuje vývojářům implementovat multiplatformní funkcionality postupně, bez nutnosti implementovat 
v Kotlinu celé vrstvy aplikací. Díky tomu dává vývojářům možnost sdílet napříč platformami je ty části kódu, které mají největších
smysl implementovat pro veškeré platformy a zbylé části kódu psát v nativním jazyce pro danou platformu. \cite{KMP}

Na obrázku \ref{fig:KMP_vrstvy} jsou vidět různé možnosti, jakými může být KMP na daných platformách implementován. 

%Jeho hlavním úkolem je sdílení kódu na mobilních platformách. 

\begin{figure}[H]
  \centering
  \includegraphics[width=1\textwidth]{KMP_vrstvy.png}
  \caption{Možnosti implementace KMP}
  \label{fig:KMP_vrstvy}
\end{figure}



\begin{figure}[H]
  \centering
  \includegraphics[width=1\textwidth]{kotlin-multiplatform-hierarchical-structure.png}
  \caption{Hierarchická struktura KMP}
  \label{fig:KMP_struktura}
\end{figure}

\subsubsection*{Expected a actual deklarace}
Deklarace Expected a actual umožňují přistupovat k platformě specifickým API z Kotlin Multiplatform modulů.


\subsubsection*{Aktuální použití KMP v praxi}
% Mezi další důvody patří taktéž lepší zastupitelnost jednotlivých členů týmu a to díky me


\emph{"Od této doby jsme vyvinuli a v produkci provozujeme několik mobilních aplikací. Ze zkušenosti vidíme, že při jejich vývoji dokážeme sdílet cca 60–70 \% kódu na platformu. V případě vývoje na dvě platformy to znamená, že v součtu dokážeme ušetřit minimálně třetinu nákladů na vývoj, plus s tím související další náklady na věci typu testování (v tomto případě snížení až na polovinu)."}

\begin{figure}[H]
  \centering
  \includegraphics[width=.7\textwidth]{chart-KMP-vs-native.png}
  \caption{Množství kódu KMP vs native}
  \label{fig:KMP_vs_native}
\end{figure}

\section{Flutter vs React Native vs Compose Multiplatform}


\subsection{Porovnání výkonu}

Mezi klíčové parametry, kvůli kterým většina firem upřednostňuje vývoj nativních aplikací, patří jednoznačně výkon
nativních aplikací. Z toho důvodu se následující podkapitola věnuje právě porovnání výkonu a dalších parametrů s
nativními verzemi aplikací. 


\subsection*{Ukázková aplikace}

Pro porovnání důležitých parametrů byla pro test vytvořena jednoduchá aplikace, která obsahovala jednu obrazovku, 
načetla obrázky z veřejného API a zobrazila je v horizontálním seznamu. 
Na každý obrázek šlo kliknout a zobrazit jej přiblížený pod seznamem. 
Pro testování byly použity toolkity pro tvorbu nativního UI pro platformu Android (Jetpack Compose) a iOS (SwiftUI)
v porovnání s multiplatformními technologiemi Compose Multiplatform a Flutter.

Mezi hlavní testované parametry patřil čas nastartování testované aplikace a její velikost.

\myparagraph{Velikost aplikace}

Na obrázku \ref{fig:chart_app_sizes} je vidět, že velikost aplikace založené na Compose Multiplatform je identická
s velikostí nativní aplikace pro Android. Je tomu tak z toho důvodu, že výsledná aplikace pro Android neobsahuje kód 
pro jiné platformy. Kdežto u velikosti iOS aplikace je situace výrazně jiná. Samotná aplikace pro iOS je v porovnání
s její nativní aplikací o 23,1 MB větší. Tento rozdíl ve velikosti je způsoben především grafickou 2D knihovnou Skia,
která je na platformě Android dostupná, kdežto na platformě iOS ne a proto tam musí být spolu s aplikací dodána.

Zajímavé je následně porovnání s velikostí aplikace založené na frameworku Flutter, jelikož ta, stejně jako Compose
Multiplatform využívá knihovnu Skia, ale i přesto je o něco menší než Compose Multiplatform aplikace na platformě iOS.
Flutter Engine je součástí aplikace spolu s kódem Dart a zvětšuje velikost (asi 3–4 MB pro Android a 10 MB pro iOS).t

\begin{figure}[H]
  \centering
  \includegraphics[width=.7\textwidth]{chart_app_sizes.png}
  \caption{APK/IPA size in megabytes}
  \label{fig:chart_app_sizes}
\end{figure}

\myparagraph{Rychlost spuštění aplikace}

Při porovnání rychlosti spuštění Compose Multiplatform aplikace na platformě Android není mezi rychlostmi téměř žádný
rozdíl stejně tak jako tomu při porovnání velikosti aplikací v předchozí kapitole. O něco delší dobu spuštění měla
aplikace napsaná pomocí frameworku Flutter, která se spouštěla v průměru o 221 ms déle než Compose Multiplatform aplikace. 
Toto zpomalení je s nejvyšší pravděpodobností způsobeno dobou spuštění Flutter Enginu, což by korespondovalo s oficiální
Flutter dokumentací. \cite{flutterPerformance}

U posledního porovnání na platformě iOS se doba spuštění aplikací založených na Compose Multiplatform a frameworku 
Flutter o tolik nelišila od nativní aplikace, což ukaje slibné použití těchto multiplatformních technologií v budoucnu.
% // TODO posledni cast porovnani

\begin{figure}[H]
  \centering
  \includegraphics[width=.7\textwidth]{chart_startup_times.png}
  \caption{App startup time on a Pixel 4a and iPhone 12 Mini in milliseconds}
  \label{fig:chart_startup_times}
\end{figure}

\section{Náročnost implementace}
\section{Limitace Compose Multiplatform oproti nativnímu řešení} 
Jedním z největších problému multiplatformního vývoje je především nedostatečné množství knihoven. 

Co se týče UI, tak zde je situace díky možnosti využití Jetpack Compose o něco lepší. Většina omezení, která jsou
aktuálně spjata s UI se týká především plynulosti a nebo použití nativně vypadajících komponent na platformě iOS.
Ta v její nativní podobě používá styl zvaný Cupertino, který aktuálně není oficiálně podporován. 


\begin{figure}[H]
    \centering
    \includegraphics[width=.7\textwidth]{composeIOS.png}
    \caption{Compose Multiplatform iOS}
    \label{fig:composeIOS}
\end{figure}

\chapter{Návrh}



\section{Výběr aplikace pro implementaci}
Základním požadavkem bylo vybrat takovou aplikaci, ve které by bylo možné použít velké množství různých komponent, 
které by bylo možné použít napříč různými platformami.
Mezi další požadavky patřilo zaměření se na kritické problémy související s multiplatformním vývojem. 
například využití 

Z těchto důvodů byla k implementaci vybrána aplikace sloužící pro občany měst či obcí, která kombinuje veškeré funkční
požadavky které plynou ze zadání.

Tato aplikace by sloužila občanům k získání informací o aktuálních novinkách, pořádaných akcích nebo jim umožňovala
zaplatit parkovné.



\section{Návrh UI}

\begin{figure}[H]
    \minipage{0.4\textwidth}
      \includegraphics[width=\linewidth]{screen1.png}
      \caption{Screen 1}\label{fig:screen1}
    \endminipage\hfill
    \minipage{0.4\textwidth}
      \includegraphics[width=\linewidth]{screen2.png}
      \caption{Screen 2}\label{fig:screen2}
    \endminipage\hfill
\end{figure}

\begin{figure}[H]
    \minipage{0.4\textwidth}
    \includegraphics[width=\linewidth]{screen3.png}
    \caption{Screen 3}\label{fig:screen3}
  \endminipage\hfill
  \minipage{0.4\textwidth}
    \includegraphics[width=\linewidth]{screen4.png}
    \caption{Screen 4}\label{fig:screen4}
  \endminipage\hfill
\end{figure}

\section{Návrh architektury}

\begin{figure}[H]
  \centering
  \includegraphics[width=1\textwidth]{mvvm.png}
  \caption{MVVM with Clean Architecture}
  \label{fig:mvvm}
\end{figure}

\dirtree{%
        .1 src \DTcomment{}.
        .2 androidMain \DTcomment{}.
        .2 commonMain \DTcomment{je pro kód sdílený mezi všemi platformami}.
        .3 kotlin \DTcomment{}.
        .4 data\DTcomment{datová vrstva}.
		.5 model\DTcomment{}.
		.5 repository.
		.4 ui\DTcomment{prezentační vrstva}.
		.5 composables\DTcomment{}.
		.5 screens\DTcomment{}.
		.6 HomeScreen\DTcomment{}.
        .3 resourses \DTcomment{}.
        .2 desktop Main \DTcomment{}.
	}




\chapter{Implementace}

\section{Řešení problémů spojených s multiplatformním vývojem}
\section{Navigace a lokalizace v implementaci}

\chapter{Testování}

\section{Možnosti testování UI v Compose Multiplatform}
\section{Implementace testů}
\section{Zhodnocení výsledků testování}

\section{Zhodnocení použitelnosti}

\chapter{Závěr}

\section{Evaluace vlastností Compose Multiplatform ve srovnání s cíli práce}
\section{Zkušenosti z implementace na reálné aplikaci}
\section{Závěr o použitelnosti frameworku}

\section{Shrnutí dosažených výsledků}
\section{Zhodnocení splnění cílů práce}
\section{Návrhy pro budoucí vývoj a výzkum}


